
\section*{Background}

MASs (multi-agent systems) is an important research topic in the field
of AI (artificial intelligence) as they consist of several intelligent
agents interacting and cooperating to reach a specific goal. This
lends itself well to distributed computing, following the recent (since
the turn of the millenium) trend of designing processors with more
cores rather than better ones. Multi-Agent Systems have also been
used to simulate naturally occurring systems, where several autonomous
agents interact. 

There have been a lot of research on the topic of MASs, as well as
related ones such as distributed artificial intelligence (DAI). There
are several APLs (Agent Programming Languages) tailored to MAS development;
they are called MASPLs (Multi-Agent System Programming Languages). 


\section*{Motivation}

While there exists many multi-agent programming languages, there are
few tools for constructing environments for the agents to behave in,
which also allows for easy graphical representation. 

There are many complications when developing multi agent systems,
our goal with this project was to lessen one of these by designing
an engine with the specific purpose to develop multi agent environments.
What these environments can be is left to the developer, however almost
everything in the engine we propose is modular and interchangeable,
ensuring that all types of multi agent environments are possible. 

What the types of projects can be is manyfold but here are some possible
examples:


\paragraph*{Agent comparison software}

There are many different languages in which it is possible to write
agent programs; some are specifically designed for it, others are
powerful enough to accommodate the possibility of agent programming.
Our engine is designed with support for multiple languages at once
which makes this engine a suitable candidate for designing a comparator
program. 

For instance, if two groups wanted to test their agent programs against
each other, this engine would make it possible for them to easily
design a world in which this test could occur.


\paragraph*{Agent testing/Simulation software}

Testing agent software can be complicated. Being able to rapidly create
an environment and visualize it can be important to a MAS project,
as it ensure basic mistakes are ironed out before larger scale implementation. 


\paragraph*{Agent teaching tools}

Teaching agent languages can be tough without proper exercises; however,
the time spent on designing these exercises can prove too exhausting
for the teacher to develop. Using our engine the teacher can rapidly
design the world he had in mind for his exercise instead of designing
every integral part of the multi agent system himself. This is because
our engine provides all the basic features of a multi agent system,
so that the time can be spent more productively on designing how a
given exercise should play out, showcasing the problem the students
are supposed to deal with.


\paragraph*{Computer games}

In practice most computer games are just multi agent programs where
one of the agents is controlled by the player. Our engine should make
it fairly easy by setting up framework for creating rules inside a
given world and ensure that the agents of the world follow said rules,
thus defining a game which the engine would be capable of running.


\section*{Goals}

The engine which we propose must reach a list of goals in order for
us to deem the project succesful:
\begin{description}
\item [{Generality:}] By this, we mean that components developed with our
engine should be \emph{reusable} in the sense that they should be
able to be used in other projects. For examle, the logic for connecting
to a specific APL should be implamentable in other projects. Furthermore,
the engine itself should be multi- applicable; it should be able to
be used to construct any kind of scenario, and interface with different
APLs.
\item [{Easy~to~use:}] It should be relatively easy to construct a complicated
scenario, and even easier to construct a simple one.
\item [{Cross~platform~compatibilty:}] The engine should be executable
on as many platforms as possible, at least the three major operating
systems (Linux, Mac and Windows).
\end{description}

\section*{Overview of the Report}

In this report we will document our efforts to design an engine as
described above. The report is structured as follows: 
\begin{description}
\item [{Theory}] outlining the design patterns used in the engine along
with explanation of MAS.
\item [{Reference~Implementation}] providing an introduction to how a
final product of the engine might look, along with agent programs
connected to the engine.
\item [{System~Features}] describing the features of the engine and how
it can be used.
\item [{Implementation}] containing an examination of the actual implementation
of the engine, along with our considerations of the choices taken.
\item [{Testing}] describes how the engine was tested.
\item [{Results~and~Comparisons}] evaluates whether the goals we had
set for the engine was reached, and compares it to similar works.\end{description}

