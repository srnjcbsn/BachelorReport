
\section{Factory Design Pattern\label{sec:TheoryFactoryDesignPattern}}

Removing flexibility from a program is normally considered a poor
design decision. One way to do that however would be by tying object
creation to the business logic of the program. This section will show
why this combining of object creation and business logic actually
removes flexibility in the program. Furthermore it will show how to
solve this problem using the Factory Design Pattern. We refer to ~\cite{WikiFactory}
for an explanation of the subject.


\subsection*{Why object creation logic should be removed from business logic}

In all OOP%
\footnote{Object Oriented Programming%
} languages you have the ability to create new objects to be used in
the program. When instantiating an object in a function, you inadvertently
tie a specific object type to that function. This means that the function
can never be used for other purposes than to work with that type of
object. Thus if an identical function was needed but for another version
of that object you would have to copy the function redundantly, greatly
increasing the chance of errors in the code as the same code was written
twice.

To give an example of this, imagine you have a Printer that prints
text on sheet of paper, the pseudo code for such a printer would look
like this:

\begin{alltt}
Class Printer 
    Method PrintPaper takes Message returns Paper
        Paper = new A4Sheet() 		
        Paper.PrintText(Message) 		
        Return Paper 	
    endMethod 
endClass
\end{alltt}

As we can see in this example, \texttt{A4Sheet} is an implementation
of the object type \texttt{Paper}, however as we have mixed object
creation with business logic, we are forced to specify the exact type
of paper that our printer produces. This means that if we wanted to
make the printer able to print multiple types of paper, we would have
to make new functions copying the functionality of \texttt{PrintPaper}
we would have to make \texttt{PrintPaperA3}, \texttt{PrintPaperA5},
etc. As is evident, this is very redundant and increases code complexity.
Not only that, but if years later someone invented a new type of paper,
the printer would have to be completely changed, since the class was
locked to specific paper types on the business logic level. 


\subsection*{Solving the problem of mixing object creation with business logic}

As we have shown, there are many problems involved when business logic
contains object creation logic. In order to solve this problem it
is necessary to separate the two. This can be done in multiple ways.
Consider our previous printer example. In this case, instead of having
the method \texttt{PrintPaper }take only a message, it could also
take paper as part of its arguments. This way we would avoid having
to create the Paper as part of printing it. However, this would only
move the problem from the \texttt{PrintPaper} method to the business
logic that is calling it. It also changes the overall functionality
of the method which was to produce paper with messages on them and
instead makes it so it doesn\textquoteright{}t produce the paper and
only prints on paper given to it. Using the factory design pattern
we can avoid both problems while at the same time separate the object
creation away from the \texttt{PrintPaper }method. We do this by giving
the \texttt{Printer }class a factory which we will name \texttt{PaperFactory}.

The \texttt{PaperFactory} only has one task and that is to create
Paper objects. That means we have moved the object creation code away
from the \texttt{PrintPaper }method; all the \texttt{PrintPaper }method
has to do is simply request a new piece of paper from the \texttt{PaperFactory}.

The pseudo code for the \texttt{PaperFactory }could look something
like this:

\begin{alltt}
Class PaperFactory 	
    Method CreatePaper takes nothing returns Paper
        Return new A4() 
    endMethod 
endClass
\end{alltt}

And the pseudo code for the new \texttt{PrintPaper} method of the
\texttt{Printer }class would be this:

\begin{alltt}
Method PrintPaper takes paperFactory, message returns paper
    Paper = paperFactory.CreatePaper()
    Paper.Print(message) 	
    Return Paper 
endMethod	
\end{alltt}

Thus as we can see the responsibility of creating paper has been removed
from the \texttt{Printer }class and instead moved to the \texttt{PaperFactory
}class.


\subsection{Abstract Factory Design Pattern}

The abstract factory design pattern is closely tied to Factory design
pattern in that an abstract Factory like an abstract class only defines
specification of the implementing class, and doesn\textquoteright{}t
contain any logic at all. 

An Abstract Factory is made by making an Abstract class of the factory.
Going back to Printer example, imagine that the \texttt{PaperFactory
}instead was an abstract class, as displayed below:

\begin{alltt}
Abstract class PaperFactory 	
    Method CreatePaper takes nothing returns Paper 
endClass
\end{alltt}

Now the Printer class only has to know about a Factory capable of
producing Paper, but it will have no information on what kind of paper
is produced, an implementation of the \texttt{PaperFactory }could
now be made for each type of paper that one wishes to produce.

For instance an implementation of the \texttt{PaperFactory} for creating
A4 papers could be designed like this:

\begin{alltt}
Class A4PaperFactory implements PaperFactory 	
    Method CreatePaper takes nothing returns paper 
        Return new A4Paper() 	
    endMethod 
endClass
\end{alltt}

As we can see, the \texttt{A4PaperFactory }class -- which is an implementation
of the \texttt{PaperFactory }class -- is capable of producing a special
type of paper. As such if this factory was used by the \texttt{Printer
}class we saw in an earlier example, the \texttt{Printer }would be
capable of producing A4 papers with messages on them. However were
we to want another type of paper, the code for the printer would not
need to be changed or copied since we simply implement a new version
of the \texttt{PaperFactory }class and give that to the \texttt{Paper
}class.


\section*{Summary}

A lot of problems arise in code when mixing business logic with object
creation logic, this section has explained why the problems occur
and what the reasons behind them are. Furthermore it has shown how
to solve these problems by using the factory design pattern. 

To summarize some of the problems when mixing business logic with
object creation logic:
\begin{itemize}
\item Increases the complexity in business logic, because of added creation
logic
\item Removes flexibility of the business logic, forcing it to work only
with special classes
\item Increase the need for redundancy of business logic to accommodate
for new object types
\item Makes changing legacy code difficult, as the code is tied to specific
object types\end{itemize}

