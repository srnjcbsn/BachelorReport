In this section we will discuss the major considerations we faced
during the project, as well as the choices we took in accomplishing
our goals, and how we could have otherwise reached them.


\section{Generality of the engine\label{sec:DiscussionGenerality}}

One of the major goals of this project was to make the engine as general
as possible. This includes the ability for the designer to implement
any kind of environment, displayed any way he wants, and controlled
by whatever APL he would like to use. In this section we will discuss
how these three vital parts of our engine lives up to this goal.

For the engine to be general means it has the ability to adapt to
any needed situation. The only restriction is that these situations
are based on a Multi agent Environment, other than that nearly any
situation should be coverable by the engine. For instance if one wishes
to use the engine to make a computer game, then the view must be able
to support a graphical display and the world of the engine must have
the ability to be changed to a 3d-world. But if instead one wishes
to make an engine for searching documents for spelling errors, then
the world should be extendable to a text-document. Furthermore to
be general also means that the engine should be used to work with
any other APL, so regardless if the APL is goal, Jason or F\# the
engine should have the ability to be adapted into working with either
of those languages.

In many cases, the shortest path to a general system is removing restrictions.
Unfortunately, features are often restrictive in nature; the most
general system of all is one that is completely featureless. Thus,
it is often a trade-off between features and generality. In our engine
design idea we try to overcome this by making all features as extensions
or optional thus not restricting the core engine.


\subsection*{Model}

For the model to be general, it should be capable of representing
a world as any possible data structure. Additionally, the objects
inhabiting it should be as general as possible, allowing them to be
defined in a way that makes sense in the context of the world. 

We have accomplished this by imposing as few restrictions as possible
on these objects. For example, as described in section \ref{sec:SystemFeaturesVirtualWorld},
a world in the engine has no data associated with it by default, leaving
the modelling of it completely in the hands of the designer. The only
restriction on the world is the idea that all entities should have
a postition in it, although the position object is completely general
itself.

Initially, we toyed with the idea of equipping the world with a graph
as the default representation of the environment, since it is a very
general data structure in the sense that it can be used to describe
other data structures. The problem with this approach is that a graph
may not be the best representation of any given world. In the case
of a tile based world, for example, a two dimensional array is more
feasible, since this is its natural representation. Ultimately, we
chose to impose no restrictions on the data structure used, and instead
rely on the use of extensions to model environments.


\subsection*{Interfacing with APLs}

One of the major problems in designing an interface that works with
different APLs is that: the order in which they, queue actions and
queries percepts may be different from APL to APL. In effect, they
do not share a common execution protocol. This means that we cannot
provide a general method for communicating with any AP. Instead, as
with other parts of the engine, the intent is to allow for extensions
capaple of interfacing with different APLs in any way they see fit.

It could be argued that using the notions of percepts and actions
serves to limit the universality of the engine. These are, however,
general concepts for interacting with an intelligent agent. They are
basically the input and output of the agent; it perceives the state
of the world, and produces an action based on this information. As
such, they are essential to interacting with an agent, and incorporated
in all agent programming languages we are aware of.


\subsection*{View}

For the view to be considered general it is paramount that the design
of the view is not being restricted in anyway, this is done by keeping
anything in the view very minimalistic. By minimalistic we mean that
the view only provides about four classes and they only provide a
tiny portion of business logic. If we had narrowed down how exactly
a view should be designed such as requiring a frame for which the
view is projected on. This might have made implementations of the
view easier as tools to draw on frames could be pre-implemented into
the engine, but in turn restricted the view from being able to become
other types of view. We did not want to do this since we think that
restrictions should be non-existent .However this also poses a potential
problem in that it is so minimalistic that we barely provide anything
for the user, and leave the user to the task of making the view by
themself.


\subsection*{Solving the Problems of Generality}

As evident when discussing how to make the engine general and how
to make it work with as many situations as possible, the problem arises
that the workload for the user gets increased. This is because whenever
we remove something from the engine in order to ensure that we impose
no restrictions. We run the risk of removing something that made the
life of the user easier, as they would not have to re-implement it
themself. To combat this problem we moved everything that added value
to the engine but imposed a restriction to the Engine Extensions project.
The idea would be that while the extensions was not part of the core
engine, they would be part of what we delivered with the engine. We
saw this as the best of both worlds, not only do we ensure that the
engine is not being restricted, but at the same time if the user did
not mind some restrictions, then they would be able to find a suitable
extension among the ones we provide. As of now the only extensions
we have are those needed for the reference implementation, but our
long term plan would be to add more extensions if possible.


\section{Model View Controller Design Pattern}

The model view controller design pattern is one of the older design
patterns within software design; its purpose is to ensure that the
developer does not deal with multiple issues at once, and instead
is able to focus on one task of the project at a time. We chose to
base our engine on the MVC pattern because we also do not want the
user of the engine to be tasked with multiple issues at once. Without
the MVC pattern, the user could be confused about how for instance
they should design a controller for an APL, and perhaps they would
mistakenly design it tailored to specific actions. If the user did
this, they would have to write new actions to perform the same task
for each new APL they encountered, which would increase code redundancy.
As the developers of this engine we wanted to ensure those kinds of
mistakes do not happen. The way we enforce this is by forcing the
MVC pattern. Since by forcing the MVC pattern we force the user to
think about how they should construct the implementation of the engine\textquoteright{}s
abstract classes. However since it is only a pattern, the user can
still make bad design decisions as we impose no restrictions. 


\section{Choice of Technologies}


\subsection*{The XMAS Engine}

We have chosen to implement the engine in C\#, which runs on the .NET
platform. While Java has a strong presence in multi-agent system development
-- as it is used by established APLs such as Jason, as well as the
EIS standard -- we have a subjective preference for C\#. In general,
C\# is a newer and more modern programming language with better facilities
for writing comprehensive and maintainable code, and provides some
features usually only found in functional programming languages. Additionally,
.NET code can be executed across platforms, thanks to the Mono project%
\footnote{\texttt{http://www.mono-project.com}%
}, although the newest version of .NET (v4.5) has not been ported at
the time of this writing. Although developing our reference implementation
would have been simpler had we used Java, opting out of this in favor
of C\# gave us the opportunity to test how well our engine interfaced
with programs not written in the same language. 


\subsection*{Reference Implementation}

As our reference implementation was developed to showcase and test
our engine, we aimed to implement it using the most commonly used
agent programming language. Since EIS can be interfaced with many
different APLs, this seemed like the obvious choice. If the engine
could be shown to work with EIS, any APL supported by EIS would work
by extension. Initially, we considered writing a J\# (.NET bindings
for the Java Language) module, which would work natively with both
our engine written in C\#, and the EIS implementation written in Java.
However, we felt that this would remove the difficulties of communicating
with an entirely different platform. This difficulty is important
to face, since 


\section{Comparison to other Environment Construction Tools\label{sec:DiscussionComparison}}

In this section, we will compare the XMAS engine to other frameworks
that can be used to construct and manage MAS environments. In particular,
we will consider CArtAgO%
\footnote{\texttt{http://cartago.sourceforge.net/}%
} (\emph{C}ommon ``\emph{Art}ifacts for \emph{Ag}ents'' \emph{O}pen
framework, henceforth referred to as Cartago), which have been used
in several projects, and as a part of the \emph{JaCaMo}%
\footnote{\texttt{http://jacamo.sourceforge.net/}%
} (\emph{Ja}son, \emph{Ca}rtago, \emph{Mo}ise) project, which provides
a complete framework for multi-agent systems consisting of the Jason
MAPL (multi-agent programming language), the Cartago environment constructions
API, and the Moise organizational system. We will also compare our
engine to using plain EIS.


\subsection{Cartago}


\subsubsection*{Agents and Artifacts}

The Cartago framework uses the \texttt{A\&A} (Agents and Artifacts)
approach to designing environments. In the following, we will provide
an overview of this model, which is presented in ~\cite{Ricci08}.

In the A\&A meta-model, a collection of computational entities called
\emph{artifacts} constitutes the environment. They are computational
entities in the sense that they are meant to provide functionality
exploitable by the agents, and can have a state and business logic,
but are not meant to act autonomously. In fact, instead of agents
having predefined actions that manipulates the state of the world
when executed, they are aware of a collection of artifacts, which
each provide a number of operations the agents can perform on them.
The artifacts also provides percepts to the agents, and are as such
the agents' means of interacting with the world. 

Artifacts can not only be used to model objects with a physical presence
in the world (such as analogues to the packages and dropzones in our
reference implementation), but also more abstract concepts, such as
control flow objects. For example, a communication artifact could
be created, through which several agents could talk and listen, through
operations and percepts, respectively. Since agents can create and
destroy artifacts at will, such communication channels are easy to
spawn in an ad hoc manner.

The A\&A meta-model, as described in ~\cite{Ricci08}, is to some
extent based on the way humans interact in a working environments,
as it draws on research in fields such as organisational sciences
and anthropology. This lead to the introduction of artifacts as tools,
service providers and communication devices, since they better describe
such an environment. In general, the A\&A approach focuses on incorporating
these concepts as an integral part of the environment, so as to make
it a functional and reactive part of a multi-agent system. This is
in contrast to classical MAS engineering, where the environment is
typically defined as a more static structure can act in and retreive
percepts from.


\subsubsection*{Cartago Implementation}

The Cartago framework is implemented in Java and can natively connect
to the Jason APL. Here, we will explain how Cartago implements the
A\&A meta-model. A more thorough explanation can be found in ~\cite{Ricci11}.
\begin{description}
\item [{Agents}] In Cartago, agent programs are connected to agent bodies,
which are -- in that respect -- conceptually similar to agents in
the XMAS engine, as they represent a vessel for the agent in the environment,
but no agent logic. In keeping with the A\&A approach explained above,
the agent API allows for creating and deleting artifacts, as well
as executing operations on artifacts and retreiving percepts from
them. 
\item [{Sensing}] For handling perception, Cartago provides the concept
of \emph{sensors}. An agent contains a set of sensors, each collecting
percepts from an artifact. The \texttt{sense} method of an agent takes
a sensor as input and returns a percept gathered by it, whereafter
the percept is removed from the sensor. The sensors can be overridden
by the designer to -- for example -- control in what order it should
return its contained percepts. A sensor is connected to an artifact
via the \texttt{focus} method. Sensory inputs can be filtered, so
that a sensor only picks up percepts matching a user-defined pattern.
\item [{Artifacts}] Artifacts specify operations that agents can execute
as described by the A\&A meta-model. Artifacts can generate events,
which can be gathered by any connected agent sensors. Artifacts can
describe how they are meant to be used, ie.\ what operations thay
have and how they should be called. When an agent executes an operation
on an artifact, a boolean value is immediately returned, signifying
either success or failure. The calling agent can give a sensor as
an argument to the method invocation, which will gather any percepts
that the artifact generates as a result of executing the operation.
\end{description}

\subsubsection*{Agents and Artifacts in XMAS}

The main purpose of the Cartago project is to provide a framework
for designing MASs using the A\&A meta-model. While that have not
been the goal of the XMAS engine, it is general enough to support
this approach, especially since the engine allows entities to incorporate
state and business logic through entity modules. Below, we have described
how artifacts, agents and perception would be implemented using the
engine:
\begin{description}
\item [{Agents}] would be XMAS agents, with a module for creating, destroying
and containing (references to) artifacts, as well as a module for
each sensor. The \texttt{sense} action used in the Cartago API is
not entirely equivalent to our generic \texttt{getAllPercepts} action,
as it only retrieves one percept from one specific sensor, but such
an action could easily be implemented. 
\item [{Artifacts}] would be represented as entities with a module for
each operation the artifact provides. When new stimuli, ie.\ new
percepts, would be available, an XMAS event would be raised.
\item [{Sensing}] The agents' sensor modules would be connected to the
artifact entities by registering triggers on them, wchich would subscribe
to the events raised by the artifacts. By using trigger conditions
(cf. section \ref{sec:SystenFeaturesEventAndTriggers}), the percepts
could be filtered as with Cartago sensors. The modules in XMAS already
provides means for being queried for collections of percepts, so this
functionality could be used to let them return all the percepts that
have become available to them since the last invocation. Alternatively,
they could incorporate some logic for the ordering of returned percepts,
in case the user only wants one percept per \texttt{sense}.
\end{description}
The functionality described above could be encapsulated in an engine
extension, providing the proper modules, events and actions. One issue
with this is that entities in the XMAS engine are meant to have a
position. As mentioned, this is not a strict requirement, as the position
can be set to a null value and ignored. Additionally, recent versions
of Cartago supports what they call workspaces, which serves to group
the agents and artifacts together in different sub-environments, for
which positions in the XMAS engine would be well suited.


\subsection{Environment Interface Standard}

While Cartago provides a very specific approach to designing an environment,
EIS is at the completely opposite side of the spectrum. The only thing
EIS provides is an interface between a single APL and Java, this means
that its goals are different from ours. What EIS is good at, is providing
a way to comunicate with a language that otherwise had no way of doing
so. While the goal of our engine is simply to use APLs that are capable
of comunication (some of them being so through EIS).

As such, comparing our engine to EIS would be a mistake since they
are different at the core, meaning that EIS is much better at what
it is designed for, while our engine is much better at what we designed
it for.
