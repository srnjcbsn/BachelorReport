In this section we will discuss the major considerations we faced
during the project, as well as the choices we took in accomplishing
our goals, and how we could have otherwise reached them.


\section{Generality of the engine}

One of the major goals of this project was to make the engine as general
as possible. This includes the ability for the designer to implement
any kind of environment, displayed any way he wants, and controlled
by whatever APL he would like to use. In this section we will discuss
how these three vital parts of our engine lives up to this goal.

For the engine to be general means it has the ability to adapt to
any needed situation. The only restriction is that these situations
are based on Multi agent Environment, other than that nearly any situation
should be coverable by the engine. For instance if one wishes to use
the engine to make a computer game, then the view must be able to
support a graphical display and the world of the engine must have
the ability to be changed to a 3d-world. But if instead one wishes
to make an engine for searching documents for spelling errors, then
the world should be extendable to a text-document. Furthermore to
be general also means that the engine should be used to work with
any other APL, so regardless if the APL is goal, Jason or F\# the
engine should have the ability to be adapted into working with either
of those languages.

In many cases, the shortest path to a general system is removing restrictions.
Unfortunately, features are often restrictive in nature; the most
general system of all is one that is completely featureless. Thus,
it is often �a trade-off between features and generality.


\subsection*{Model}

For the model to be general, it should be capable of representing
a world as any possible data structure. Additionally, the objects
inhabiting it should be as general as possible, allowing them to be
defined in a way that makes sense in the context of the world. 

We have accomplished this by imposing as few restrictions as possible
on these objects. For example, as described in section \texttt{\emph{{[}System
Features.World{]}}}, a world in the engine has no data associated
with it by default, leaving the modelling of it completely in the
hands of the designer. The only restriction on the world is the idea
that all entities should have a postition in it, although the position
object is completely general itself. \texttt{\emph{{[}Environments
where postition does not matter?{]}}}

Initially, we toyed with the idea of equipping the world with a graph
as the default representation of the environment, since it is a very
general data structure in the sense that it can be used to describe
other data structures. The problem with this approach is that a graph
may not be the best representation of any given world. In the case
of a tile based world, for example, a two dimensional array is more
feasible, since this is its natural representation. Ultimately, we
chose to impose no restrictions on the data structure used, and instead
rely on the use of extensions to model environments.


\subsection*{Interfacing with APLs}

One of the major problems in designing an interface that works with
different APLs is that the order in which they queue actions and queries
percepts may be different from APL to APL. In effect, they do not
share a common execution protocol. This means that we cannot provide
a general method for communicating with any AP. Instead, as with other
parts of the engine, the intent is to allow for extensions capaple
of interfacing with different APLs in any way they see fit.

It could be argued that using the notions of percepts and actions
serves to limit the universality of the engine. These are, however,
general concepts for interacting with an intelligent agent. They are
basically the input and output of the agent; it perceives the state
of the world, and produces an action based on this information. \texttt{\emph{{[}Explain
why the functions of an agent can not be reduced any further (why
they are the basis for an agent), refer to AIMA{]}}} As such, they
are essential to interacting with an agent, and incorporated in all
agent programming languages we are aware of.


\subsection*{View}

For the view to be considered general it is paramount that the design
of the view is not being restricted in anyway, this is done by keeping
anything in the view very minimalistic. By minimalistic we mean that
the view only provides about four classes and they only provide a
little about of business logic. If we had narrowed down how exactly
a view should be designed such as requiring a frame for which the
view is projected on. This might have made implementations of the
view easier as tools to draw on frames could be pre-implemented into
the engine. We did not want to do this since we think that restrictions
should be non-existent .However this also poses a potential problem
in that it is so minimalistic that we barely provide anything for
the user, and leave the user to the task of making the view by themself.


\subsection*{Solving the Problems of Generality}

As evident when discussing how to make the engine general and how
to make it work with as many situations as possible, the problem arises
that the workload for the user gets increased. This is because whenever
we remove something from the engine in order to ensure that we impose
no restrictions, we run the risk of removing something that made the
life of the user easier, since they would not have to re-implement
it themself. To combat this problem we moved everything that added
value to the engine but imposed a restriction to the Engine Extensions
project. The idea would be that while the extensions was not part
of the core engine, they would be part of what we delivered with the
engine. We saw this as the best of both worlds, not only do we ensure
that the engine is not being restricted, but at the same time if the
user did not mind some restrictions, then they would be able to find
a suitable extension among the ones we provide. As of now the only
extensions we have are those needed for the reference implementation,
but our long term plan would be to add more extensions if possible.


\section{Model View Controller Design Pattern}

The model view controller design pattern is one of the older design
patterns within software design; its purpose is to ensure that the
developer does not deal with multiple issues at once, and instead
is able to focus on one task of the project at a time. We chose to
base our engine on the MVC pattern because we also do not want the
user of the engine to be tasked with multiple issues at once. Without
the MVC pattern, the user could be confused about how for instance
they should design a controller for an APL, and perhaps they would
mistakenly design it tailored to specific actions. If the user did
this, they would have to write new actions to perform the same task
for each new APL they encountered, which would increase code redundancy.
As the designers of this engine we wanted to ensure those kinds of
mistakes do not happen. The way we enforce this is by forcing the
MVC pattern. Since by forcing the MVC pattern we force the user to
think about how they should construct the implementation of the engine\textquoteright{}s
abstract classes. However since it is only a pattern, the user can
still make bad design decisions as we impose no restrictions. As most
bad design decisions comes from making an easy implementation of something,
we hopefully reduce the number of times they are made by making bad
decisions harder. \texttt{\emph{{[}rewrite sentence?{]}}}


\section{Test Driven Development}


\section{Choice of Technologies}


\subsection*{The XMAS Engine}

We have chosen to implement the engine in C\#, which runs on the .NET
platform. While Java has a strong presence in multi-agent system development
-- as it is used by established APLs such as Jason, as well as the
EIS standard -- we have a subjective preference for C\#. In general,
C\# is a newer and more modern programming language with better facilities
for writing comprehensive and maintainable code, and provides some
features usually only found in functional programming languages. Additionally,
.NET code can be executed across platforms, thanks to the Mono project
\texttt{\emph{{[}Link to Mono{]}}}, although the newest version of
.NET (v4.5) has not been ported at the time of this writing. Although
developing our reference implementation would have been simpler had
we used Java, opting out of this in favor of C\# gave us the opportunity
to test how well our engine interfaced with programs not written in
the same language. 


\subsection*{Reference Implementation}

As our reference implementation was developed to showcase and test
our engine, we aimed to implement it using the most commonly used
agent programming language. Since EIS can be interfaced with many
different APLs, this seemed like the obvious choice. If the engine
could be shown to work with EIS, any APL supported by EIS would work
by extension. Initially, we considered writing a J\# (.NET bindings
for the Java Language) module, which would work natively with both
our engine written in C\#, and the EIS implementation written in Java.
However, we felt that this would remove the difficulties of communicating
with an entirely different platform. This difficulty is important
to face, since 


\section{Comparison to other Environment Construction Tools}

Comparison to JaCaMo, EIS
