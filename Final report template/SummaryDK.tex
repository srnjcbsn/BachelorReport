\chapter{Summary (Danish)}
\begin{otherlanguage}{danish}

I denne rapport vil vi dokumentere vores bestræbelser på at udvikle Xmas (cross platform Multi-Agent System) maskinen, der er designet til at udvikle MAS (Multi-Agent System) miljøer og holde styr på intelligente agenter, der agerer heri. Da maskinen bruges til at lave miljøer, er det meningen at agenterne sender sanselige indtryk til og modtager kommandoer fra et separat agent-programmeringssprog, der implementerer agenternes kunstige intelligens.

Det primære mål med projektet er at gøre maskinen så generel som muligt, således at ethvert ønskeligt miljø kan udvikles med den, og samtidig gøre det nemt at udvide individuelle komponenter, så de passer til specifikke MAS-typer. Maskinen kommer med indbygget understøttelse for sammenkobling med EIS (Environment Interface Standard), og dermed også for sammenkokbling med de af EIS understøttede agent-programmeringssprog. Et simpelt flise-baseret miljø følger også med maskinen. Maskinen er designet ud fra model-view-controller udviklingsmønsteret, hvilket muliggør en klar opdeling af de forskellige komponenter. For at teste og fremvise vores maskine har vi udviklet en referenceimplementation der bruger agent-programmeringssproget GOAL til at styre dens agenter. 

Vi mener at Xmas i høj grad er generel, hvilket har betydet at flere funktionaliteter er blevet rykket fra Xmas-kernen ud til udvidelsespakker. Xmas er bedst egnet til større systemer, da der skal en del kode til at designe, opsætte og eksekvere sytemer med den. Både Xmas og udvidelsespakkerne kører på de dominerende operativsystemer, Linux, Windows og Mac OS inkluderet.

\end{otherlanguage}