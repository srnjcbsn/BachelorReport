
\section{Architecture}

The section will cover how all the components of the engine interacts
with one another, it will detail how flow of information is transferred
through the engine and into the components connected to it. The component
diagram of the Xmas Engine can be found in appendix \ref{ComponentDiagramAppendix}.

The components of the engine are
\begin{itemize}
\item Model
\item World Creation
\item View
\item Controller
\end{itemize}

\subsection{Model Component}
\begin{description}
\item [{Requires:}] \texttt{XmasWorld}, \texttt{XmasAction }and \texttt{Trigger }
\item [{Provides:}] \texttt{Percept}
\end{description}
The model component is responsible for handling internal interactions
of the engine. These interactions are based on which \texttt{XmasAction
}it is given. 

The model component has three requirements, these requirements are
necessary for the model component to properly execute the environment
requested by the user.

The first requirement of the model component is the \texttt{XmasWorld},
the model component uses the \texttt{XmasWorld }by giving it to \texttt{ActionManager},
the \texttt{ActionManager }then gives the \texttt{XmasWorld }to all
\texttt{XmasActions }as they are about to be executed. 

The second requirement is the \texttt{XmasActions}, all \texttt{XmasActions
}queued on the model component are executed by the \texttt{ActionManager}.
An \texttt{XmasAction }are not executed immediately however, as they
wait until all prior \texttt{XmasActions} executions has been completed.
Once queued to the \texttt{ActionManager}, they are provided with
all necessary dependencies such as the \texttt{XmasWorld }and the
\texttt{EventManager}.

\texttt{XmasAction }is designed to allow other threads the ability
to interact with the engine. The reason is that we did not wish for
multiple threads to change the state of the model component at once
is that would force the designer using the engine to deal with mutli
threading problems. To guarentee that this is never necessary we provide
the abiltity to inject code into the \texttt{ModelComponent} thread,
which is transferred in the form of an \texttt{XmasAction}.

The last requirement of the model component is \texttt{Trigger}, the
model component takes any number of triggers and inserts them in the
\texttt{EventManager}. When an \texttt{XmasAction }raise\textquoteright{}s
an \texttt{XmasEvent }on the \texttt{EventManager}, the \texttt{Triggers
}that are registered to that \texttt{XmasEvent }are all triggered. 

The only thing that the model component provides is the \texttt{Percept},
each \texttt{Percept }is something that an agent can sense. An \texttt{AgentController
}connected to the model component can receive these \texttt{Percept}s\texttt{
}which it is meant to use for analyzing the agent\textquoteright{}s
next move.

The model component is made of many classes however the three \texttt{XmasModel},
\texttt{EventManager }and \texttt{ActionManager }are what provide
the core features of the model component and as such is the only ones
shown in the diagram. When going into details of the exact design
of the engine it will be evident that the class \texttt{XmasEntity
}also provides some of the features of both \texttt{EventManager }and
\texttt{ActionManager}, however it only does this to make the feel
of using the engine more natural. For instance when moving an entity
we thought that it would make sense that the code for this was \texttt{Entity.QueueAction(new
Move())}, instead of \texttt{ActionManager.QueueAction(new Move(EntityToBeMoved))}.
In actuality the code does the same thing since in the first case:
All the \texttt{Entity }does is to call the \texttt{ActionManager}
in the same way we just showed and then use itself in place of the
\texttt{EntityToBeMoved}. This is the reason why the \texttt{Entity
}is not shown in the model component as it has no relevance when understanding
the component itself.


\subsection{World Creation Component}
\begin{description}
\item [{Provides:}] \texttt{XmasWorld}
\end{description}
The world creation component is responsible for making a world for
the engine\textquoteright{}s entities to inhabit. The world is created
when the engine starts to execute, as such its internal class \texttt{WorldBuilder
}only contains a blue print for which entities it should construct
and not the actual entities themselves. It does this by storing a
function for each entity, those functions contains the information
on how each of the entities should be constructed.

The user of the engine is meant to implement his own \texttt{WorldBuilder
}class, that implementation should contain knowledge on how the world
he wishes to construct is created. That means if for instance wants
to use a Tile based world then his implementation of \texttt{WorldBuilder}
should construct a tile based world.


\subsection{View Component}
\begin{description}
\item [{Provides:}] \texttt{Trigger}
\end{description}
The view component is meant as the component that visualizes the model
of the engine to the user, it does not enforce how the visualization
is done or in which way the visualization occurs. It only provides
the tools necessary to perform this task.

The view is meant to register \texttt{Triggers }on the model component,
these \texttt{Triggers }contains \texttt{XmasEvents }when an \texttt{XmasEvent
}is raised, the \texttt{Triggers }with those \texttt{XmasEvents }are
triggered. The idea is that when a \texttt{Trigger }is triggered that
means the current state of the model component has changed, the view
uses these \texttt{Triggers }to be informed about such changes, and
are thus able to change its own state in responds correctly making
it able to visualize the new model state. 


\subsection{Controller Component}
\begin{description}
\item [{Requires:}] \texttt{Percept }
\item [{Provides:}] \texttt{XmasAction}
\end{description}
The controller component\textquoteright{}s responsibility is to command
\texttt{Agents }to perform actions inside the world. The controller
component does this by making the \texttt{AgentController }send \texttt{XmasAction
}objects to a specific \texttt{Agent }in the model component. Where
upon that \texttt{Agent }will perform said \texttt{XmasAction}, once
the model component has executed all prior actions.

The controller component also has ability to receive \texttt{Percept
}objects back from the engine, these \texttt{Percept }contain data
about what the \texttt{Agent }it is controlling has sensed. These
\texttt{Percepts }are meant to be analyzed by the controller component
to determine what its next \texttt{XmasAction }should be. 

The controller component is made up of abstract classes which the
user of the engine must first implement; these implementations could
be setup to act as an interface between a single APL and our engine.
This means that for each APL one must make a new implementation of
the controller component. To reduce the burden of the user we will
in our extensions provide the ability to interface with EIS supported
APLs.

Furthermore the controller component is not only designed to make
interfacing with different APLs easier, it is also meant to be used
when making an interface between the user and the model component.
For instance if one wished to control an agent with the keyboard,
then an Keyboard implementation of the AgentController and AgentManager
should be made, where it would be possible to bind the queuing of
move actions to specific buttons on the keyboard, this done as part
of our Reference implemenation that can be seen in its source code. 


\subsection*{Summary}

The architecture of the engine shows the connectivity between each
of the components. The Model component which job it is to ensure proper
interactions occur inside the world. The world which is constructed
by the World Creation Component, meant to be designed along with the
world itself. 

The interactions of the model component are provides by the controller
component which task it is to command the agents inside the engine,
and make it so they are given intelligence. And lastly the view component
which only task is to visualize the state of the engine.
