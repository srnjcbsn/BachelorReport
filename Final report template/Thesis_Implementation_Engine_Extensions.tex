
\section{Engine Extensions}


\subsection{Tile Extension}


\subsubsection*{Vision}

As the basic rules for which tiles can be seen from where have been
explained in the System Features section, we will now turn our attention
to how corners are connected with each other. 

Recall that a corner $c_{1}$ on a tile $t_{1}$ connects to a corner
$c_{2}$ of another tile $t_{2}$ if a straight line can be traced
from $c_{1}$ to $c_{2}$ without intersecting with a tile that is
blocking the line. In the tile extension, a tile is blocking the line
if it contains an entity that is movement blocking with respect to
the entity looking from $t_{1}$. Additionally, the tile $t_{2}$
is visible from $t_{1}$ if at least one corner of $t_{1}$ connects
to at least three corners of $t_{2}$. If $t_{2}$ is vision blocking,
only two corners of $t_{2}$ need be connected to. In its most simple
form, the algorithm iterates over all the tiles in the agent's visible
range, and returns a collection containing just those satisfying the
above condition. 

The interesting part of the algorithm is this: how do we find all
the tiles a line from one corner to another passes through? \texttt{\emph{{[}Figure{]}}}
To accomplish this, we find the slope of the line as $\left|\frac{v_{x}}{v_{y}}\right|$,
where $v$ is the vector describing the line. If either $v_{x}$ or
$v_{y}$ is zero, the line only passes between tiles, and it is handled
as a special case.

\texttt{\emph{{[}Explain that it is similar to that other algorithm
(name?){]}}}
