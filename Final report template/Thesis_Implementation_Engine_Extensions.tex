
\section{Engine Extensions}


\subsection{Tile Extension\label{sub:ImplementationTileExtension}}

The tiled environment is quite elementary, as it is basically a two-dimensional
array with some extra arithmetic to change the coordinate system,
as described in System Features. The actions and events are also very
straight-forward, as they function as described in their own sections.
The most interesting part of the tile extension is thus the way we
handle vision, which we will describe here.


\subsubsection*{Vision}

We say that the tile $t_{2}$ is visible from another tile $t_{1}$
if at least one corner of $t_{1}$ connects to at least three corners
of $t_{2}$. If $t_{2}$ is vision blocking, only two corners of $t_{2}$
need be connected to. In figure \ref{fig:TileMapVision} we have shown
some examples of connecting corners. Next, we will explain what it
means for two corners to connect.

\begin{figure}
\begin{centering}
\includegraphics[width=0.5\textwidth]{tilemapVision}
\par\end{centering}

\caption{In this figure we see the visibility of selected tiles $\mathtt{T}_{1}\dots\mathtt{T}_{4}$with
regards to the agent in position $\mathtt{A}$. The light gray squares
represent vision blocking tiles, and all other tiles are see-through.
Lines have been drawn from the corners of the agent's tile to the
corners of the selected tiles. Green lines signifies that the two
corners connect, whereas red lines means they are not connected. Tiles
$\mathtt{T}_{1}$, $\mathtt{T}_{2}$ and $\mathtt{T}_{4}$ are visible
to the agent, while $\mathtt{T}_{3}$ is obscured.\label{fig:TileMapVision}}


\end{figure}


We say that a corner $c_{1}$ on a tile $t_{1}$ connects to a corner
$c_{2}$ of another tile $t_{2}$ if a straight line can be traced
from $c_{1}$ to $c_{2}$ without intersecting with a tile that is
blocking the line. In the tile extension, a tile is blocking the line
if it contains an entity that is vision blocking with respect to the
entity looking from $t_{1}$. This presents a problem with a line
whose vector has an $x$ or $y$ component equal to 0. As such a line
never intersects with any tiles (it only passes between them), it
will always connect with the rules stated above, even if all tiles
it passes are vision blocking. Thus, we say that a line which only
extends in the $x$ or $y$ direction does not connect if it passes
between two vision blocking tiles.

Note that in fig. \ref{fig:TileMapVision}, the line from the SE corner
of \texttt{A} to the SW corner of $\mathtt{T}_{3}$ does not connect,
as it passes between two vision blocking tiles, as described above.
Also note that $\mathtt{T}_{4}$ should, intuitively, be obscured
to the agent; we should have a rule saying that when a line intersects
with a corner, it should be blocked if both of the two tiles that
are connected to the corner, and is not intersected by the line, are
vision blocking.

In its most simple form, the vision algorithm iterates over all the
tiles in the agent's visible range, and returns a collection containing
just those satisfying the above condition, as shown in the pseudo
code below:

\begin{alltt}
\textit{Method} Vision 
        \textit{takes} an Agent A, 
        \textit{returns} a collection of Tiles
    Tiles : Collection of Tiles
    for each Tile T in A's visible range:
        \textit{if} T.isVisionBlocking(A):
            \textit{if} any one corner of A connects with any two corners of T:
                add T to Tiles
        \textit{else}:
            \textit{if} any one corner of A connects with any three corners of T:
                add T to Tiles
    \textit{return} T
\end{alltt}

\begin{figure}
\begin{centering}
\includegraphics[width=0.4\textwidth]{tilemapConnection}
\par\end{centering}

\caption{A line described by the vector $v=(3,4)$ with slope $s=\frac{v_{y}}{v_{x}}=\frac{4}{3}$.
Squares in light gray are tiles the line intersect with.\label{fig:ConnectingLine}}
\end{figure}


The interesting part of the algorithm is this: how do we determine
whether two corners connect? That is, how do we find all the tiles
a line from one corner to another passes through? 

To illustrate the problem, consider the line depicted in fig. \ref{fig:ConnectingLine}.
First, we find the slope of the line as $s=\frac{v_{y}}{v_{x}}$,
where $v$ is the vector describing the line. If we consider the line's
equation ($y=s\cdot x$), as it is depicted in the figure, we can
see that the line segment decribed by the vector between the points
$\left(x_{0},s\cdot x_{0}\right)$ and $\left(x_{0}+1,s\cdot\left(x_{0}+1\right)\right)$
(where $x_{0}\in\mathbb{N}$) crosses the tiles between $x=x_{0}$
and $x=x_{0}+1$ on the $x$-axis and $y=\left\lfloor s\cdot x_{0}\right\rfloor $
and $y=\left\lceil s\cdot\left(x_{0}+1\right)\right\rceil $ on the
$y$ axis. This can be repeated for every $x_{0}$ in the range $0\dots v_{x}$
to obtain the complete collection of intersected tiles.

In the \texttt{Vision} class, the \texttt{walkAlongVector(Vector v)}
method performs the above operations, with some modifications to handle
vectors with negative components properly. It is called by the \texttt{ConnectCorners(Point,
Point)} method, which calculates the vector \texttt{v}. \texttt{WalkAlongVector}
uses the \texttt{yield return} keyword to return each tile at a time,
so lines are only drawn until a blocking one is encountered.


\subsection{EIS Extension}

EIS support in engine is provided with a special \texttt{AgentController
}and\texttt{ AgentManager }class, along with a specially designed
java EIS environment jar file. This section will go through how the
implementation works and how we connect to the EIS environment.

The EIS environment in java and the agent controller on the engine
is connected through a TCP connection. They communicate with each
other with XML as a markup language for the data they transmit. 

Fig. \ref{fig:DeploymentEISandAgentController} shows the setup between
EIS and the agent controller.

\begin{figure}
\begin{centering}
\includegraphics[width=1\textwidth]{DeploymentEISandAgentController}
\par\end{centering}

\caption{An illustration of the connection between the EIS environment and
the agent controller.\label{fig:DeploymentEISandAgentController}}


\end{figure}


Although the EIS environment and the agent controller sends all data
in form of XML data there is one difference and that is all XML nodes
are packaged into packages of a certain size and the size is sent
before the xml data, as can be seen in fig. \ref{fig:DataPackaging}.

\begin{figure}
\begin{centering}
\includegraphics[scale=0.8]{XMLDataPackageFigure}
\par\end{centering}

\caption{Image of the data sent between the EIS environment and the agent controller.\label{fig:DataPackaging}}


\end{figure}


This was done to ensure that the agent controller at all times knew
how much data was to be transmitted, thus giving it the right to deny
packages if they were over a certain size. In our current implementation
however no package size is denied. 


\subsubsection*{Engine Side of EIS Support}

In the project: \texttt{XmasEngineExtensions }we provide the following
two classes:
\begin{description}
\item [{\texttt{EISAgentController:}}] this class is responsible for converting
xml data from the EIS environment into actions that can be queued
to the agent it controls. And also for converting percepts the agent
it controls receives into XML data that can be sent to the EIS environment.
\item [{\texttt{EISAgentServer}}] creates a TCP server. All EIS environments
that wish to connect to it must make a TCP client call. Once a connection
is established, the agent server will construct an \texttt{EISAgentController
}object, that object will take over all further duties of comunication.
\end{description}

\subsubsection*{How the EISAgentServer works}

The server manages the agent controllers and it also manages the connection
creation between an EIS environment and an \texttt{EisAgentController}.

In Fig. \ref{fig:EISServerSequenceDiagram} we show how an EIS environment
connects to an agent server and how the agent server handles the connection.
The connection works by the EIS environment making a TCP connection
request. The agent server then responds by constructing the agent
controller (and give it its own thread). Once the agent controller
is constructed the agent server is no longer responsible for handling
that connection and leaves it up to the agent controller to find out
what the EIS environment wants. This is basically to connect to a
given agent whom it knows by name, and start sending it actions.

\begin{figure}
\begin{centering}
\includegraphics[width=1\textwidth]{EISServerSequenceDiagram}
\par\end{centering}

\caption{A sequence diagram of an EIS environment connecting to the engine
through an EisAgentServer.\label{fig:EISServerSequenceDiagram}}


\end{figure}



\subsubsection*{How the EisAgentController works}

The EIS agent controller's job is to ensure that all demands made
by an EIS environment are met. This is done by receiving actions in
XML data form and convert the data into Xmas Actions. These actions
are then queued onto an agent.

\begin{figure}
\begin{centering}
\includegraphics[width=1\textwidth]{EisAgentControllerSequenceDiagram}
\par\end{centering}

\caption{A sequence diagram showing how XML data from EIS environment are converted
into Xmas data.\label{fig:EisAgentControllerSequenceDiagram}}


\end{figure}


In fig. \ref{fig:EisAgentControllerSequenceDiagram}, it is shown
how xml data received is converted by the controller and then sent
to the agent inside the engine. Percepts are only sent if they are
updated. In this case the action was to retrieve percepts. If the
action had instead been to move an agent, then no percepts would be
sent by the controller.


\subsubsection*{EIS Environment}

The EIS environment is designed to setup an interface between an APL
and an environment. The way we use EIS is by making it a hollow link
between our engine and the APL (such as GOAL). Thus the EIS environment
implementation we make must be able to provide communication between
the APL and our engine. The way we have done this is by making the
environment convert all the XML data it receives from Xmas into IILang
objects, which is an object tree implemented in EIS for recursively
representing percepts and actions in native Java code as well as converting
them to and from XML and prolog statements. It can then use the IILang
objects in its internal GOAL interface.

\begin{figure}
\begin{centering}
\includegraphics[width=1\textwidth]{EISEnvironmentToGoalSequenceDiagram}
\par\end{centering}

\caption{This diagram shows how the communication between goal and the EIS
environment works.\label{fig:EISEnvironmentToGoalSequenceDiagram}}


\end{figure}


An example of sequence for the EIS environment communicating with
an APL (such as GOAL) can be seen on fig. \ref{fig:EISEnvironmentToGoalSequenceDiagram}.
The basic idea is that the goal program sends commands directly to
the EIS environment we have implemented and then we ensure that those
commands are fulfilled by transmitting them over to the \texttt{EisAgentController}
through a TCP connection.


\subsection*{Considerations}

The design of interfacing with GOAL was originally what the engine
design was mostly focused on; as such there have been lots of different
approaches to this interfacing that we have gone through. One approach
was to connect the EIS environment using J\# which could be converted
into C\# byte code; this would be a lot faster than our current approach
since XML data wastes a lot of space by encapsulating every bit of
information. However J\# is an old language and we wanted to ensure
that we did not run into too many complications under development
as such we chose our current approach since the real time transmission
of data is not as important as the idea of it, for this project in
particular. 


\subsection*{Summary}

EIS is an interface for designing environments in java that connects
to EIS supported APLs, we use this environment to develop an environment
that is simply a TCP connection between the APL(in our case goal)
and our engine. The design provides the necessary features to the
engine, but the design could have been more optimized by using a more
compact way of sending data, since sending data as XML nodes takes
up a lot of space since XML requires all data to be encapsulated by
it.
