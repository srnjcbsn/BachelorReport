
\section{Overview}

Before we begin explaining all the features of the engine, we would
like to point out that if it is necessary for the reader to see an
actual implementation using all the feature, there is one available
in appendix \ref{chap:VacuumWorldAppendix} (Vacuum World).

The goal of the engine is to allow for simulation of a world where
agents within are allowed to act. As such, it is important that it
can accurately model a state-machine. To model a state-machine, one
must have the ability to contain a state and perform actions to change
the current state. 

A complete UML Domain model diagram is provided in appendix \ref{sec:Domain-Model-UML}.


\subsection{State}

In our domain model, we have state stored in three object types:
\begin{itemize}
\item World
\item Entities
\item Modules
\end{itemize}

\paragraph*{World}

The world is the place all entities are meant to inhabit as either
agents of the world or simply objects for other entities to interact
with. The world is not defined by the engine. As shown in fig. \ref{fig:DomainModelDiagramXMAS},
it is an abstract class, meaning it is the developer using our engine
that defines the world. As such the world can be any type of world
needed, it could be a 3-d world, a 2-d world, a world based on tiles
or hexagons, or simply be nodes with an undefined number of edges
connecting each other.


\paragraph*{Entities}

The world is empty without anything inside it, as such we have the
entities which are meant to model the objects one would have the world
contain. For example, in our reference implementation, we have a world
with packages scattered about a maze. It is then the task of the agents
to collect these packages; the entities here are not only the walls
of the maze and the packages, but also the agents since they inhabit
the world as well. The agents are different from entities in that
they all have a name. This name is unique and is meant to be a way
of distinguishing the agents from one another. 


\paragraph*{Modules}

The modules can be viewed as either the constraints or as the abilities
of all entities. For example, if you wanted to constrain entities
from moving into each other, you would create a \emph{movement blocking
module}, which would contain information on whether or not a given
entities is allowed to pass through it. However, if you wanted to
give an agent the ability to move, a \emph{speed module} would be
required. Whether a certain module is a constraint or an ability is
up to the individual module.


\subsection{Actions}

A world is static and unexciting if one is not allowed to perform
any changes to it, for this we provide what we have chosen to name
actions. There are two different types of actions: environment actions
and entity actions. The core difference between them is that entity
actions are meant as actions performed by a single entity, such as
moving the entity or having the entity pick up another object. Environment
actions are actions that affect the entire world. In our domain model,
we have chosen to add two actions that are built into the engine,
the first is an entity action that gets all the percepts for a given
entity called \texttt{GetallPerceptsAction} and the other is an environment
action that can shut down the engine called \texttt{CloseEngineAction}. 


\subsection{Events and Triggers}

The engine relies heavily upon events, this means that all actions
performed within the engine is meant to trigger events in responds.
This can be used to either activate new actions within the engine,
or be meant to transfer data to the views listening. 

In order to listen to the events, a trigger need to be created with
all the events it listens to registered to it. Furthermore, a trigger
needs a condition and an action. The condition is a predicate that
determines whether the trigger is fired, and the action is the function
that is excuted when the trigger is fired.
