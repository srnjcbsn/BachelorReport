In this project, we have developed an engine for constructing scenarios
to be used in multi-agent systems. A scenario in this sense is an
environment where agents can interact, and with means to control the
agents. Our project provides the following features:
\begin{itemize}
\item A way to set up environments.
\item Means of communicating with APLs, which controls the agents inhabiting
the environment.
\item Provide a general way for agents to behave in the environment, by
allowing them to \emph{act} and \emph{sense}.
\item Reactivity in the environments, allowing agents and entities to dynamically
respond to changes in the world.
\end{itemize}

\section{Results of comparisons}

As discussed in \ref{sec:DiscussionComparison}, our engine is placed
somewhere in between Cartgo and EIS in terms of multi-applicability
and features. That is, our engine is more general than Cartago, while
providing more convenient features for a MAS than EIS (and, conversely,
being more specific in its implementation than EIS, and less feature
complete than Cartago). We have argued that the meta-model used in
the Cartago project can be implemented in Xmas, and shown that EIS
can be used as an extension to the engine in order to exploit its
APL compatibility.


\section{Engine completion}

We will now evaluate whether the goals we listed in the introduction
have been reached:
\begin{description}
\item [{Generality:}] As we discussed in section \ref{sec:DiscussionGenerality},
we believe we have reached a good amount of generality in the engine,
as we impose few restrictions on the design of environments. The eternal
problem is the trade-off between generality and features, and much
of our effort have been directed towards equalizing these two qualities.
Without several implementations of the engine, it is difficult to
say how well our solution caters to different environment systems
and their needs.
\item [{Ease~of~use:}] While our engine features tools that can ease
the development of larger systems, it is not comparatively easier
to use in small scale apllications. Our minimal example implementation
(Vacuum World, see Appendix \ref{chap:VacuumWorldAppendix}) consists
of a total of 17 C\# classes. It should be noted, however, that the
example does not use any extensions apart from the very basic logger
extension. Extensions can be used to define some of the abstract notions
such as position, and thus ease the development burden. It is also
worth noting that most of the classes contains very little code.
\item [{Cross~platform~compatibility:}] We have compiled the project
against the mono platform, which provides the .NET platform on Mac,
Linux and Windows. We have developed the engine in both Linux and
Windows, and it works as it should on both platforms. At the moment,
our reference implementation can not be run on Linux or Mac due to
a bug in Mono regarding text buffer sizes when printing to a terminal.
While the reference implementation is an important part of this project,
we do not consider it a part of the core of the Xmas engine, and therefore
conclude that this goal has been reached.
\end{description}
In summation, we believe that the goals for the engine have mostly
been reached.

However, There are many aspects of the engine which could be severely
improved. These include, but is not limited to, 

\textbullet{} The specific way actions and events are designed. For
instance, the distinction between environment events and entity events,
as discussed in section \ref{sub:ImplementationEventsTriggers}.

\textbullet{} The way the view is supposed to communicate with the
model

\textbullet{} The way agent controllers are forced to have their own
threads. This is unnecessary when APLs runs all agents on a single
thread, however inefficient that may be.


\section{Future work}

Here, we will list some of the possible additions to the engine, which
would make it easier to use by providing functionality for a wide
array of MAS scenarios.
\begin{itemize}
\item Extensions to communicate with other APLs, that do not support the
EIS standard.
\item A collection of common environments, such as:

\begin{itemize}
\item Three dimensional worlds
\item A graph based world
\end{itemize}
\item An extension allowing construction and management of discrete worlds.
\item Extending the reference implementation to include agents cooperating
to find packages. 
\item Copying the environment and functionality of the Agents on Mars scenario,
to showcase and test our engine against an established implementation.
\item A collection of commonly used agent actions and percepts, such as
means of communicating with each other.\end{itemize}

