
\subsection{Overview}

The goal of the engine is to allow for simulation of a world where
agents within are allowed to act. As such, it is important that it
can accurately model a state-machine. To model a state-machine, one
must have the ability to save a state and perform actions to change
the current state. A complete UML Domain model diagram is provided
in appendix \ref{sec:Domain-Model-UML}.


\subsubsection{State}

In our domain model, we have state stored in three object types:
\begin{itemize}
\item World
\item Entities
\item Modules
\end{itemize}

\paragraph*{World}

The world is the place all entities are meant to inhabit as either
agents of the world or simply objects for the entities to interact
with. The world is not defined by the engine. As shown in fig. \ref{fig:DomainModelDiagramXMAS},
it is an abstract class, meaning it is the developer using our engine
that defines the world. As such the world can be any type of world
needed, it could be a 3-d world, a 2-d world, a world based on tiles
or hexagons or simply be nodes with an undefined number of edges connecting
each other.


\paragraph*{Entities}

The world is empty without anything inside it, as such we have the
entities which are meant to model the objects one would have the world
contain. For example in our reference implementation using our engine,
we have a world with packages scattered about a maze. It is then the
task of the agents to collect these packages; the entities here are
not only the walls of the maze and the packages, but also the agents
since they inhabit the world as well. The agents are different from
entities in that they all have a name. This name is unique and is
meant to be a way of distinguishing the agents from one another. 


\paragraph*{Modules}

The modules can be viewed as the constraints and as the abilities
of all entities. For instance if you wanted to constrain entities
from moving into each other than you would create a Movement blocking
module, this module would then contain information on whether or not
a given entities is allowed to pass through it. \texttt{\emph{{[}Rewrite?{]}}}


\subsubsection{Actions}

A world is static and unexciting if one is not allowed to perform
any changes to it, for this we have what we have chosen to name actions.
There are two different types of actions: environment actions entity
actions. The core difference between them is that entity actions are
meant as actions performed by a single entity, such as moving the
entity or having the entity pick up another object. Environment actions
are actions that affect the entire world. In our domain model, we
have chosen to add two actions that are built into the engine, the
first is an entity action that gets all the percepts for a given entity
called \texttt{GetallPerceptsAction} and the other is an environment
action that can shut down the engine called \texttt{CloseEngineAction}. 


\subsubsection{Events and Triggers}

The engine relies heavily upon events, this means that all actions
performed within the engine is meant to trigger events in responds.
This can be used to either activate new actions within the engine,
or be meant to transfer data to the views listening. 

In order to listen to the events, a trigger need to be created with
all the events it listens to registered to it. Furthermore, a trigger
needs a condition and an action. The condition is a predicate that
determines whether the trigger is fired, and the action is the function
that is excuted when the trigger is fired.
